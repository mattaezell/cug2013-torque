\documentclass[10pt, conference, compsocconf]{IEEEtran}

% Various packages that might be useful
\usepackage[pdftex]{graphicx}
\usepackage{array}
\usepackage[tight,footnotesize]{subfigure}
\usepackage{url}
\hyphenation{op-tical net-works semi-conduc-tor}
\usepackage{color}


\begin{document}

\title{Production Experiences with the Cray-Enabled TORQUE Resource Manager}

\author{\IEEEauthorblockN{Matt Ezell and Don Maxwell}
\IEEEauthorblockA{High Performance Computing Operations\\
Oak Ridge National Laboratory\\
Oak Ridge, TN\\
\{ezellma,maxwellde\}@ornl.gov}
\and
\IEEEauthorblockN{David Beer}
\IEEEauthorblockA{Senior Software Engineer\\
Adaptive Computing\\
Provo, UT\\
dbeer@adaptivecomputing.com}
}

\maketitle


\begin{abstract}
High performance computing resources utilize batch systems to manage the user
workload. Cray systems are uniquely different from typical clusters due to
Cray’s Application Level Placement Scheduler (ALPS). ALPS manages binary
transfer, job launch and monitoring, and error handling. Batch systems require
special support to integrate with ALPS using an XML protocol called BASIL.

Previous versions of Adaptive Computing’s TORQUE and Moab batch suite integrated
with ALPS from within Moab, using PERL scripts to interface with BASIL. This
would occasionally lead to problems when all the components would become
unsynchronized. Version 4.1 of the TORQUE Resource Manager introduced new
features that allow it to directly integrate with ALPS using BASIL. This paper
describes production experiences at Oak Ridge National Lab using the new TORQUE
software versions.
\end{abstract}

\begin{IEEEkeywords}
TORQUE; Resource Manager; Adaptive Computing; Cray; ALPS; Moab; HPC; Titan; Gaea
\end{IEEEkeywords}


\section{Introduction}

\section{Cray ALPS}

\cite{alps}

\section{Previous Design}

\section{New Design}
\subsection{Orphan ALPS Reservations}

\section{Production Experiences}

\subsection{Experiences on Gaea}

\subsection{Experiences on Titan}

\section{Future Work}

\section{Conclusion}

% use section* for acknowledgement
%\section*{Acknowledgment}

%\textcolor{red}{
%	The authors would like to thank...
%	more thanks here
%}

% trigger a \newpage just before the given reference
% number - used to balance the columns on the last page
% adjust value as needed - may need to be readjusted if
% the document is modified later
%\IEEEtriggeratref{8}
% The "triggered" command can be changed if desired:
%\IEEEtriggercmd{\enlargethispage{-5in}}

% references section
\bibliographystyle{IEEEtran}
\bibliography{IEEEabrv,cug2013_torque}

% that's all folks
\end{document}


