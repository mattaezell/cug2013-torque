\documentclass[10pt, conference, compsocconf]{IEEEtran}

% Various packages that might be useful
\usepackage[pdftex]{graphicx}
\usepackage{array}
%\usepackage[tight,footnotesize]{subfigure}
%\usepackage{url}
\hyphenation{op-tical net-works semi-conduc-tor}
\usepackage{color}


\begin{document}

\title{Production Experiences with the Cray-Enabled TORQUE Resource Manager}

\author{\IEEEauthorblockN{Matt Ezell and Don Maxwell}
\IEEEauthorblockA{High Performance Computing Operations\\
Oak Ridge National Laboratory\\
Oak Ridge, TN\\
\{ezellma,maxwellde\}@ornl.gov}
\and
\IEEEauthorblockN{David Beer}
\IEEEauthorblockA{Senior Software Engineer\\
Adaptive Computing\\
Provo, UT\\
dbeer@adaptivecomputing.com}
}

\maketitle


\begin{abstract}
High performance computing resources utilize batch systems to manage the user
workload. Cray systems are uniquely different from typical clusters due to
Cray’s Application Level Placement Scheduler (ALPS). ALPS manages binary
transfer, job launch and monitoring, and error handling. Batch systems require
special support to integrate with ALPS using an XML protocol called BASIL.

Previous versions of Adaptive Computing’s TORQUE and Moab batch suite integrated
with ALPS from within Moab, using PERL scripts to interface with BASIL. This
would occasionally lead to problems when all the components would become
unsynchronized. Version 4.1 of the TORQUE Resource Manager introduced new
features that allow it to directly integrate with ALPS using BASIL. This paper
describes production experiences at Oak Ridge National Lab using the new TORQUE
software versions.
\end{abstract}

\begin{IEEEkeywords}
TORQUE; Resource Manager; Adaptive Computing; Cray; ALPS; Moab; HPC; Titan; Gaea
\end{IEEEkeywords}


\section{Introduction}

High performance computing resources utilize batch systems to manage the user
workload. Job schedulers are designed to intelligently determine when jobs
should run, optimizing for various goals such as high utilization or minimal
queue wait. Resource managers typically accept job submissions and handle job
launch. Cray systems are uniquely different from typical clusters due to an
additional layer called Cray’s Application Level Placement Scheduler (ALPS).
ALPS manages binary transfer, job launch and monitoring, and error handling.
Batch systems require special support to integrate with ALPS using an XML
protocol called BASIL.

Previous versions of Adaptive Computing’s TORQUE and Moab batch suite
integrated with ALPS from within Moab, using PERL scripts to interface with
BASIL. This would occasionally lead to problems when all the components would
become unsynchronized. Additionally, TORQUE was unaware of the Cray compute
nodes. Version 4.1 of the TORQUE Resource Manager introduced new features that
allow it to directly integrate with ALPS using the BASIL protocol.

This paper describes early experiences with the newest versions of the TORQUE
resource manager. Early on, software bugs related to the newly-introduced
multithreading features prevented successful deployment of the new versions of
TORQUE. Through close collaboration with Adaptive Computing, the software
improved significantly to the point where it was acceptable for use on the
Titan and Gaea systems. Additionally, this paper describes production
experiences at Oak Ridge National Lab using the new TORQUE software versions
and describes future collaborative work to improve Cray-enabled TORQUE.


\section{Cray Application Level Placement Scheduler (ALPS)}

The term ``resource manager'' is overloaded in the batch processing world, and
the combination of systems needed to effectively control batch processing on
the Cray X-series platform certainly supports the confusion.  At the lowest
level in the Cray hierarchy sits the Cray Application Level Placement Scheduler
commonly referred to as ALPS \cite{alps}.  While in most instances, a batch
system is making scheduling decisions based upon configuration of center
policy, ALPS is the piece of software at the lowest level that is handed
information from the batch system to ultimately launch the job onto the compute
nodes.

Not only does ALPS launch jobs, it maintains compute node state and
reservations to manage job placement and resource utilization.  Through a
series of daemons that typically run on the Cray boot or system database (sdb)
node, ALPS imports hardware configuration information from the system database
to provide memory, CPU and GPU resources available on each compute node.  Given
a heterogeneous system, this would be essential to providing the user with a
mechanism to request the resources needed to run a particular application.
Each compute node also has a state and mode associated with it that informs
ALPS whether the node is up or down and whether it is in interactive or batch
mode.  Up and down states are self-explanatory, and there are a few other
states that will not be discussed that are in the end classified as up or down,
but interactive or batch mode requires some explanation.  

ALPS can operate without a batch system sitting at a higher level providing
information.  This is called interactive mode, and it is basically a FIFO queue
that requires users to run ALPS commands that sit and wait for available
resources to run jobs.  How is this different from a batch system?  In a
nutshell, batch systems provide a mechanism for the user to submit a job to be
run at a later time without the burden of making sure the machine doesn't
reboot taking the waiting ALPS command down with it, and batch systems also
allow a reordering of jobs based on policies determined by the center.  In
contrast, batch mode simply enables the use of a batch system for job launch
ignoring any user-supplied ALPS commands run outside of the batch system.

Reservations are used to manage availability of nodes that are in an up state.
When a job is launched, it is launched on a set of nodes, and those nodes are
reserved for that job.  When the job finishes, the reservation is destroyed,
and those nodes are available for the next job.  Reservations are simply the
mechanism by which a job receives exclusive access to the resources necessary
to run the job.

All of the ALPS information must somehow get communicated to the batch system
in order to provide the user with the available resources on the system and to
maintain a consistent state for nodes, reservations, and jobs.  ALPS provides
an API called \emph{apbasil} - BASIL being an acronym for Batch Application
Scheduler Interface Layer.  BASIL is an XML-based protocol that provides batch
systems with the ability to retrieve inventories of compute nodes, their states
and reservations, and the ability to create, confirm, and delete reservations.
Using this interface, batch systems are able to manage all aspects of job
submission, scheduling, placement and deletion by communicating with ALPS which
communicates directly with the compute nodes.



\input{sections/prev_design}

\section{New Design}

\subsection{Orphan ALPS Reservations}


\section{Production Experiences}

\subsection{Early Experiences on Gaea}

The National Climate-Computing Research Center (NCRC) at Oak Ridge National
Laboratory houses and operates high performance computing resources for the
National Atmospheric and Oceanic Administration (NOAA) and its research
partners.  NCRC's flagship system is named Gaea, after the Greek goddess of
earth.  Gaea was delivered and upgraded in phases, ultimately culminating in
two production partitions with a combined peak performance of 1.1 petaflops.

In July of 2012, the ``c1'' partition of Gaea was upgraded from a Cray XT6
system with the SeaStar interconnect and AMD ``Magny Cours'' processors to a
Cray XE6 system with the Gemini interconnect and AMD ``Interlagos'' processors.
The previous system used TORQUE 2.5.x.  Moab was setup in a grid; there was a
central Moab that made scheduling decisions and an instance of Moab on c1 that
interfaced with ALPS.  The hardware upgrade also required a software upgrade,
and it was determined that TORQUE version 4.1 should be installed and tested.

At the time, version 4.1.0 was the latest release available.  Several important
bugs had already been identified and fixed, so ORNL decided to pull souce from
the 4.1-fixes branch in subversion.  Installation and initial testing were
successful, but it wasn't long before problems were discovered.

The first problem encountered in the acceptance test was missing \emph{PBS_O_*}
environment variables.  The acceptance harness used variables such as
\emph{PBS_O_WORKDIR} to setup the correct environment prior to running a job.
When those were missing, the jobs would fail.

The most severe bug found early on was related to improper handling of
environment variables.  When variables were set without values, the qsub
command would segfault and fail to submit the job.  Also, variable values that
contained comma or newline characters would be incorrectly split and sent to
the environment as multiple variables.  This behavior was not obvious at first
glance, but it was causing an acceptance application to fail due to a missing
environment variable that increased the thread stack size.

TORQUE 4.0 included a major architectural change: the pbs_server was made
multi-threaded.  While this is essential for performance and scaling, it also
introduces the possibility of race conditions and deadlocks.  Although the
developers were careful in their implementation, several deadlocks were
experienced.

Another bug caused Cray jobs to fail when the pbs_server was restarted.  It was
due to Cray nodes not being present when the jobs were being recovered.  Also,
X11 forwarding in interactive jobs was not working correctly, preventing the
use of debuggers and GUI frontends.  A problem was also encountered where
trying to delete a running job in the Cray environment failed.  The server was
trying to communicate with the cray compute node instead of the login node that
was running the job script.

As each major bug was identified and fixed, ORNL pulled the latest source from
subversion and ran with that.  TORQUE version 4.1.1 was released on August 30,
2012, but there were no significant relevant fixes above the version already in
production.  In late October, the Gaea environment moved to TORQUE version
4.1.3.  This version contains several fixes, including some patches developed
at ORNL.


\subsection{Experiences on Titan}

While Jaguar never ran the new architecture aside from the beta test shot,
Titan began its life in September 2012 running the new design.  While many of
the issues inherent in a new design had already been discovered and fixed
during the beta, subsequent use on the TDS’s, and production on Gaea; the Titan
acceptance team put the new software stack through its paces at scale for true
production.  This transition also provided a new opportunity to externalize
both Moab and TORQUE servers from the Cray platform to provide job access for
users even when the Cray was unavailable.  Two fundamental changes at once is
generally not a good idea, but all in all, things have went fairly well.

Overwhelmingly, the primary problem that has been seen on Titan is deadlocks in
the TORQUE server.  Threading the version 4 TORQUE server is clearly a step in
the right direction, but it has come with some growing pains.  A deadlocked
TORQUE server causes issues for the entire batch system from simple job
submission failure to a hung or at least very slow Moab server.  Primarily due
to the fact that the end users were seeing job submission failures, a script
was created early in the acceptance period which first determined the TORQUE
server was deadlocked, gdb attached to the server and generated a core file,
and then restarted the server.  By running this script on a regular schedule
via cron, the pain felt by the users became much more bearable until the
deadlocks were found and fixed.  Through the efforts of both ORNL and Adaptive
staff, all known deadlocks have been fixed, and the batch system is running
well at this point. 

The transition to an external Moab and TORQUE server has certainly been well
received by the users.  Having the ability to manipulate jobs when the Cray is
unavailable provides the user with everything needed for the job process except
the actual execution.  Some effort had to be devoted to finding a TORQUE server
bug that prevented this from working, but that has now been fixed as well.

Again, the transition to the new architecture and the external servers has been
a success in spite of a few growing pains.  Ultimately, the users have seen
benefits from both better synchronization with ALPS and the ability to
manipulate jobs while the Cray itself is unavailable.



\input{sections/future_work}

\section{Conclusion}

TORQUE's quality has been somewhat spotty over the last year as major
architectural changes have been implemented.  Thanks to the diligent work of
the TORQUE developers and the community input, TORQUE has significantly
improved.

Cray's ALPS software is very unique, and it requires batch systems to add
special support to interoperate.  Adaptive's original design that interfaced
with ALPS from Moab was effective, but the new TORQUE-based ALPS integration is
more straightforward and higher performing.  The new software provides benefits
that are highly visible to end-users.


\section*{Acknowledgment}

The authors would like to thank the TORQUE developers as well as the TORQUE
community for constantly improving TORQUE.


% references section
\bibliographystyle{IEEEtran}
\bibliography{IEEEabrv,cug2013_torque}

% that's all folks
\end{document}


