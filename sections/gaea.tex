\subsection{Early Experiences on Gaea}

The National Climate-Computing Research Center (NCRC) at Oak Ridge National
Laboratory houses and operates high performance computing resources for the
National Atmospheric and Oceanic Administration (NOAA) and its research
partners.  NCRC's flagship system is named Gaea, after the Greek goddess of
earth.  Gaea was delivered and upgraded in phases, ultimately culminating in
two production partitions with a combined peak performance of 1.1 petaflops.

In July of 2012, the ``c1'' partition of Gaea was upgraded from a Cray XT6
system with the SeaStar interconnect and AMD ``Magny Cours'' processors to a
Cray XE6 system with the Gemini interconnect and AMD ``Interlagos'' processors.
The hardware upgrade also required a software upgrade, and it was determined
that TORQUE version 4.1 should be installed and tested.

At the time, version 4.1.0 was the latest release available.  Several important
bugs had already been identified and fixed, so ORNL decided to pull souce from
the 4.1-fixes branch in subversion.  Installation and initial testing were
successful, but it wasn't long before problems were discovered.

The first problem encountered in the acceptance test was missing \emph{PBS_O_*}
environment variables.  The acceptance harness used variables such as
\emph{PBS_O_WORKDIR} to setup the correct environment prior to running a job.
When those were missing, the jobs would fail.

The most severe bug found early on was related to improper handling of
environment variables.  When variables were set without values, the qsub
command would segfault and fail to submit the job.  Also, variable values that
contained comma or newline characters would be incorrectly split and sent to
the environment as multiple variables.  This behavior was not obvious at first
glance, but it was causing an acceptance application to fail due to a missing
environment variable that increased the thread stack size.

Another bug caused Cray jobs to fail when the pbs_server was restarted.  It was
due to Cray nodes not being present when the jobs were being recovered.  Also,
X11 forwarding in interactive jobs was not working correctly, preventing the
use of debuggers and GUI frontends.  A problem was also encountered where
trying to delete a running job in the Cray environment failed.  The server was
trying to communicate with the cray compute node instead of the login node that
was running the job script.

As each bug was identified and fixed, ORNL pulled the latest source from
subversion and ran with them.  TORQUE version 4.1.1 was released on August 30,
2012.
