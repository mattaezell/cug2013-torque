\subsection{Experiences on Titan}

While Jaguar never ran the new architecture aside from the beta test shot,
Titan began its life in September 2012 running the new design.  While many of
the issues inherent in a new design had already been discovered and fixed
during both the beta and subsequent use on the TDS’s, the Titan acceptance team
put the new software stack through its paces at scale for true production.
This transition also provided a new opportunity to externalize both Moab and
TORQUE servers from the Cray platform to provide job access for users even when
the Cray was unavailable.  Two fundamental changes at once is generally not a
good idea, but all in all, things have gone fairly well.

Overwhelmingly, the primary problem that has been seen on Titan is deadlocks in
the TORQUE server.  Threading the version 4 TORQUE server is clearly a step in
the right direction, but it has come with some growing pains.  A deadlocked
TORQUE server causes issues for the entire batch system from simple job
submission failure to a hung or at least very slow Moab server.  Primarily due
to the fact that the end users were seeing job submission failures, a script
was created early in the acceptance period which first determined the TORQUE
server was deadlocked, gdb attached to the server and generated a core file,
and then restarted the server.  By running this script on a regular schedule
via cron, the pain felt by the users became much more bearable until the
deadlocks were found and fixed.  Through the efforts of both ORNL and Adaptive
staff, all known deadlocks have been fixed, and the batch system is running
well at this point. 

The transition to an external Moab and TORQUE server has certainly been well
received by the users.  Having the ability to manipulate jobs when the Cray is
unavailable provides the user with everything needed for the job process except
the actual execution.  Some effort had to be devoted to finding a TORQUE server
bug that prevented this from working, but that has now been fixed as well.

Again, the transition to the new architecture and the external servers has been
a success in spite of a few growing pains.  Ultimately, the users have seen
benefits from both better synchronization with ALPS and the ability to
manipulate jobs while the Cray itself is unavailable.

